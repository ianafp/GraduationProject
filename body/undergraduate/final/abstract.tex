\cleardoublepage{}
\begin{center}
    \bfseries \zihao{3} 摘~要
\end{center}

随着图形处理技术的发展,GPU在处理复杂三维场景中的作用日益凸显。本研究旨在为国产GPU设计并实现一种高效的Point-in-Triangle并行测试算法,以提升其在图形渲染中的性能。由于现有的串行处理方式在处理大量数据时存在效率瓶颈,本课题通过深入分析和优化GPU光栅化模块中的 Point-in-Triangle算法,提出了一种新的并行化策略,以充分利用国产GPU的并行处理能力。

本课题研究内容涵盖了现有算法的分析、并行化策略设计、算法优化与实现、性能评估与比较以及硬件适配性研究。通过对边界方程法、面积法和射线法的复现与分析,本课题发现了并行性优化空间,并设计了适应国产GPU架构的并行Point-in-Triangle测试模型。在性能测试方面,通过比较优化前后的算法表现,验证了新算法在处理速度、资源消耗和渲染质量方面的显著提升。

此外,本课题还探讨了算法在国产GPU上的适配性和可扩展性,以及在游戏图形渲染、虚拟现实、科学可视化等实际应用中的潜力。技术文档与指导的编写,为国内开发者提供了详细的算法设计和实现细节,推动了国产GPU生态系统的成熟和发展。
\cleardoublepage{}
\begin{center}
    \bfseries \zihao{3} Abstract
\end{center}
With the advancement of graphics processing technology, the role of Graphics Processing Units (GPUs) in handling complex 3D scenes has become increasingly prominent. This study aims to design and implement an efficient Point-in-Triangle parallel testing algorithm for domestic GPUs to enhance their performance in graphic rendering. Given the efficiency bottlenecks of existing serial processing methods when dealing with large amounts of data, this research has conducted an in-depth analysis and optimization of the Point-in-Triangle algorithm within the GPU rasterization module. A new parallelization strategy has been proposed to fully leverage the parallel processing capabilities of domestic GPUs.

The research content of this topic covers the analysis of existing algorithms, the design of parallelization strategies, the optimization and implementation of algorithms, performance evaluation and comparison, and hardware adaptability research. Through the reproduction and analysis of boundary equation methods, area methods, and ray methods, this research has identified spaces for parallelism optimization and designed a parallel Point-in-Triangle testing model adapted to the architecture of domestic GPUs. In terms of performance testing, the new algorithm's significant improvements in processing speed, resource consumption, and rendering quality have been verified by comparing its performance before and after optimization.

Additionally, this research has explored the adaptability and scalability of the algorithm on domestic GPUs, as well as its potential in practical applications such as game graphic rendering, virtual reality, and scientific visualization. The compilation of technical documentation and guidance has provided domestic developers with detailed algorithm design and implementation details, promoting the maturity and development of the domestic GPU ecosystem.